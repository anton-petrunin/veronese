\documentclass[a4paper,10pt]{article}
\usepackage{paper-en}
\usepackage{hyperref}




%\usepackage[notref,notcite,color]{showkeys}



\def\thetitle{Mildly curved submanifolds in a ball}
\def\theauthors{Anton Petrunin}

\hypersetup{colorlinks=true,
citecolor=black,
linkcolor=black,
anchorcolor=black,
filecolor=black,
menucolor=black,
urlcolor=black,
pdftitle={\thetitle},
pdfauthor={\theauthors}
}








%\usepackage[a-2b,mathxmp]{pdfx}[2018/12/22]
%\overfullrule=100mm
%\usepackage[none]{hyphenat}
\begin{document}
%\pagestyle{empty}\renewcommand\includegraphics[2][{}]{}


\title{\thetitle}
\author{\theauthors}
\date{}
\maketitle

\begin{abstract}
Suppose $M$ is a closed submanifold in a Euclidean ball of large dimension.
We give an optimal bound on the normal curvatures of $M$ that guarantee that it is a sphere.
The border cases consist of Veronese embeddings of four projective planes.
\end{abstract}

\paragraph{Introduction.} Let $M\subset \RR^d$ be a closed smooth $n$-dimensional submanifold.
Assume $d$ is large and $M$ lies in an $r$-ball.
\textit{What can we say about the normal curvatures of $M$?}

First note that the curvatures cannot be smaller than $\tfrac1r$ at all points.
Moreover, 
\textit{the average value of $|H|$ must be at least $n\cdot\tfrac1r$};
here $H$ denotes the mean curvature vector \cite[28.2.5]{burago-zalgaller}, \cite[3.1]{petrunin2024a}.
This statement is a straightforward generalization of the result of István Fáry \cite{fary,tabachnikov} about closed curves in a ball.

On the other hand, the $n$-dimensional torus can be embedded into an $r$-ball with all normal curvatures $\sqrt{3\cdot n/(n+2)}\cdot\tfrac1r$.
This embedding was found by Michael Gromov
\cite[2.A]{gromov3}, \cite[1.1.A]{gromov2}.
This bound is optimal; that is, any smooth $n$-dimensional torus in an $r$-ball has normal curvature at least $\sqrt{3\cdot n/(n+2)}\cdot\tfrac1r$ at some point
\cite{petrunin2024a}.
Gromov's examples easily imply the following:
any closed smooth manifold $M$ admits a smooth embedding into an $r$-ball of sufficiently large dimension with normal curvatures less than $\sqrt{3}\cdot\tfrac1r$
\cite[1.D]{gromov3}, \cite[1.1.C]{gromov2}.
\textit{But what happens between $\tfrac1r$ and  $\sqrt{3}\cdot\tfrac1r$?}

In this note, we consider embeddings in an $r$-ball with normal curvatures at most $\tfrac2{\sqrt{3}}\cdot \tfrac1r$.
We show that if the inequality is strict, then the manifold must be homeomorphic to a sphere (see §~\ref{thm:strict}).
For the nonstrict inequality, in addition to spheres we get real, complex, quaternionic, and octonionic planes mapped by the corresponding Veronese embedding up to rescaling (see §~\ref{thm:=}).

\paragraph{Sphere theorem.}
\label{thm:strict}
\textit{Let $M$ be a closed smooth $n$-dimensional submanifold in a closed $r$-ball in $\RR^d$.
Suppose that the normal curvatures of $M$ are strictly less than $\tfrac2{\sqrt{3}}\cdot\tfrac1r$.
Then $M$ is homeomorphic to the $n$-sphere.}


\parit{Proof.}
Denote the $r$-ball by $\BB^d$.
We can assume that $r=\tfrac1{\sqrt{3}}$;
therefore, the normal curvatures of $M$ are smaller than $2$.

Choose a unit-speed geodesic $\gamma\:[0,\tfrac\pi2]\to M$;
let $x=\gamma(0)$ and $y=\gamma(\tfrac\pi2)$.
By the assumption, the curvature of $\gamma$ in $\RR^d$ is less than~$2$.
Applying Schur's bow lemma, we get $|x-y|>1$.

Let $\Pi$ be the perpendicular bisector to $[x,y]$.
Since the curvature of $\gamma$ is smaller than 2,
\[\measuredangle(\gamma'(t_0),\gamma'(t))< 2\cdot|t-t_0|,
\quad\text{and}\quad
\langle \gamma'(t_0),\gamma'(t) \rangle> \cos (2\cdot|t-t_0|)\] if $|t-t_0|>0$.
Therefore
\[\langle y-x,\gamma'(t_0) \rangle>\int\limits_0^{\frac\pi2}\cos (2\cdot |t-t_0|)\cdot dt\ge0.\]
In particular, the function $f\:[0,\tfrac\pi2]\to\RR$ defined by
$f\:t\mapsto \langle y-x,\gamma(t) \rangle$
has positive derivative.
Therefore, $\gamma$ intersects $\Pi$ transversely at a single point;
denote it by $s$.



Choose a unit vector $\mathsc{u}\in\T_x$;
let $\gamma_{\mathsc{u}}\:[0,\tfrac\pi2]\to M$ be the unit-speed geodesic that starts from $x$ in the direction ${\mathsc{u}}$, and let $z=\gamma_{\mathsc{u}}(\tfrac\pi2)$.
The argument above shows that $|x-z|>1$.

Denote by $H_x$ and $H_y$ the closed half-spaces bounded by $\Pi$ that contain $x$ and $y$ respectively.
Assume $z\in H_x$, then we have $|y-z|\ge |x-z|>1$.
Since $|x-y|>1$, the triangle $[xyz]$ has all sides larger than $1$,
which is impossible since $x,y,z\in \BB^d$.
Therefore, $\gamma_v$ meets $\Pi$ before in $\tfrac\pi2$;
denote by $r({\mathsc{u}})$ be the first such time moment.

\begin{wrapfigure}{r}{44 mm}
\vskip-0mm
\centering
\includegraphics{mppics/pic-10}
\vskip2mm
\end{wrapfigure}

Let us show that the function ${\mathsc{u}}\mapsto r({\mathsc{u}})$ is smooth.
In other words, $\gamma_{\mathsc{u}}$ intersects $\Pi$ transversely at time $r({\mathsc{u}})$.
Assume this is not the case, so $\gamma_{\mathsc{u}}$ is tangent to $\Pi$ at $r({\mathsc{u}})$.
Let $\hat\gamma_{\mathsc{u}}$ be the concatenation of the reflection of $\gamma_{\mathsc{u}}|_{[0,r({\mathsc{u}})]}$ across $\Pi$ and $\gamma_{\mathsc{u}}|_{[r({\mathsc{u}}),\frac\pi2]}$.
Note that $\hat \gamma_{\mathsc{u}}$ is $C^1$-smooth, and it is $C^\infty$-smooth everywhere except $r({\mathsc{u}})$.
Therefore, Schur's bow lemma is applicable to~$\hat \gamma_{\mathsc{u}}$, and hence, $|y-z|>1$.
Again, all sides of triangle $[xyz]$ are larger than $1$;
hence, it cannot lie in $\BB^d$ --- a contradiction.  

It follows that the set 
\[V_x=\set{t\cdot {\mathsc{u}}\in \T_x}{|{\mathsc{u}}|=1,\quad 0\le t\le r({\mathsc{u}}),}\]
is diffeomorphic to the closed $n$-disc.
Denote by $W_x$ the connected component of $x$ in $M\cap H_x$.

From the Gauss formula \cite[Lemma 5]{petrunin2024}, the sectional curvatures of $M$ are less than $4$.
In particular, the exponential map $\exp_x\:\T_x\to M$ is a local diffeomorphism in the $\tfrac\pi2$-ball centered at the origin of $\T_x$.

It follows that $\exp_x\:V_x\to W_x$ is a local diffeomorphism;
in particular, $W_x$ is a smooth manifold with boundary.
Since $V_x$ is simply connected, $\exp_x$ defines a diffeomorphism $V_x\to W_x$.
In particular, $W_x$ is a closed topological $n$-disc and $\partial W_x$ is a smooth hypersurface in $M$.

Let us swap the roles of $x$ and $y$, and repeat the construction.
We get another closed topological $n$-disc $W_y\subset M$ bounded by a smooth hypersurface $\partial W_y$.

Observe that $\partial W_x$ intersects $\partial W_y$ at $s$.
Furthermore, both $\partial W_x$ and $\partial W_y$ are connected components of $s$ in $M\cap \Pi$.
Therefore, $\partial W_x=\partial W_y$.
That is, $M$ can be obtained by gluing two $n$-discs by a diffeomorphism between their boundaries.
Hence $M$ is homeomorphic to the $n$-sphere.
\qeds

\paragraph{Veronese embeddings.}\label{thm:=}
Our next result is an application of the following theorem;
its weaker form was proved by Detlef Gromoll and Karsten Grove \cite{gromoll-grove}, and
the final step was made by Burkhard Wilking \cite{wilking}.

\begin{thm}{Gromoll--Grove--Wilking theorem}\label{thm:GGW}
Let $M$ be a compact Riemannian manifold with sectional curvature at least $1$ and
diameter at least $\tfrac\pi2$.
If $M$ is not homeomorphic to a sphere, then its Riemannian universal cover is isometric to a compact rank-one symmetric space.
\end{thm}

Recall that simply connected compact rank-one symmetric spaces include round spheres $\mathbb{S}^n$, complex and quaternionic projective spaces $\CP^n$ and $\HP^n$, and octonionic projective plane $\OP^2$.
(We will also need the real projective space $\RP^n$, which is a quotient of $\mathbb{S}^n$.)
Let us assume that each of these spaces is equipped with the canonical metric;
so spheres have constant sectional curvature $1$ and the sectional curvature of projective spaces is in the range $[1,4]$.
In particular, all the projective spaces including $\RP^n$ have closed geodesics of length~$\pi$.

\begin{thm}{Proposition}
There are smooth isometric embeddings
\begin{itemize}
 \item $\RP^n \hookrightarrow\RR^d$ for $d\ge n+\tfrac12\cdot n\cdot(n+1)$;
 \item $\CP^n \hookrightarrow\RR^d$ for $d\ge n+n\cdot(n+1)$;
 \item $\HP^n \hookrightarrow\RR^d$ for $d\ge n+2\cdot n\cdot(n+1)$;
 \item $\OP^2 \hookrightarrow\RR^d$ for $d\ge 26$;
\end{itemize}
that map each geodesic to a round circle.

In particular, all normal curvatures of the images of these embeddings are equal to $2$.
Moreover, the images of these embeddings lie in a sphere of radius $r=\sqrt{n/(2\cdot n+2)}$ (for $\OP^2$, we assume that $n=2$). 
\end{thm}

The proposition can be extracted from two theorems in \cite[§ 2]{sakamoto}.
The embeddings provided by the proposition will be called \emph{Veronese embeddings}.
From the proof below it will be clear that the properties in the proposition uniquely describe
Veronese embeddings up to motion of ambient space.

The Veronese embeddings have a very explicit algebraic description and many nice geometric properties.
In particular,
these embeddings are equivariant, and 
their images are minimal submanifolds in the described spheres.
All of this is discussed in the cited paper by Kunio Sakamoto.
 
Now we are ready to formulate our result.

\begin{thm}{Rigidity theorem}
Let $M$ be a closed smooth $n$-dimensional submanifold in a closed $r$-ball in $\RR^d$.
Suppose that the normal curvatures of $M$ are at most $\tfrac2{\sqrt{3}}\cdot\tfrac1r$.
If $M$ is not homeomorphic to a sphere, then up to rescaling, it is congruent to an image of the Veronese embedding of a projective plane $\RP^2$, $\CP^2$, $\HP^2$, or $\OP^2$.
\end{thm}

\parit{Proof.}
Assume $M$ is not homeomorphic to a sphere;
in this case, $\dim M\ge 2$.
As before, $\BB^d$ will denote the $r$-ball in $\RR^d$, and we assume that $r=\tfrac1{\sqrt{3}}$;
therefore, the normal curvatures of $M$ are at most $2$.



Choose a unit-speed geodesic $\gamma\:[0,\tfrac\pi2]\to M$;
let $x=\gamma(0)$ and $y=\gamma(\tfrac\pi2)$.
The argument in our sphere theorem implies that $|x-y|=1$.
The rigidity case in the bow lemma implies that $\gamma$ is a half-circle of curvature $2$.
Since any two points in $M$ can be connected by a geodesic, we get the following.
\begin{itemize}
 \item The diameter of $M$ is 1.
 \item The intrinsic diameter and injectivity radius of $M$ are equal to $\tfrac\pi2$.
 \item All geodesics in $M$ are circles of curvature 2 in $\RR^d$.
\end{itemize}

Furthermore, for $x$ and $y$ as above,
there is another point $z\in M$ such that $|x-z|=|y-z|=1$.
If not, then again, the argument in the sphere theorem would imply that $M$ is a sphere.
But since $x,y,z\in\BB^d$,
the equalities $|x-y|=|y-z|=|x-z|=1$ imply that $x\in \partial \BB^d$.

The choice of $x\in M$ was arbitrary.
Therefore, $M$ lies in the sphere $\partial \BB^d$ of radius $r=1/\sqrt{3}$.
This sphere has sectional curvature $1/r^2=3$;
the normal curvatures of $M$ in the sphere are $\kappa=\sqrt{2^2-1/r^2}=1$.
By the Gauss formula \cite[Lemma 5]{petrunin2024}, the sectional curvatures of $M$ are at least $3-2\cdot \kappa^2=1$.

By the Gromoll--Grove--Wilking theorem, the universal cover $\tilde M$ of $M$ is isometric to a rank-one symmetric space.
Taking into account the injectivity radius and curvature of $M$, we get that $\tilde M$ must be isometric to one of the following spaces
$\tfrac12\cdot \SSS^n$, $\SSS^n$, $\CP^n$, $\HP^n$ for some $n$, or $\OP^2$.
Note that the points $x,y,z\in M$ constructed above lie at an intrinsic distance $\tfrac\pi2$ from each other.
It forbids $\tfrac12\cdot \SSS^n$ for every $n$.
Furthermore if $n\ge 3$, then each space  $\SSS^n$, $\CP^n$ and $\HP^n$ contain 4 points at a distance $\tfrac\pi2$ from each other.
Since the injectivity radius of $M$ is $\tfrac\pi2$, their projections in $M$ must lie at a distance $\tfrac\pi2$ from each other as well.
It follows that $\BB^d$ must contain 4 points at a distance 1 from each other, which is impossible.

Hence, $\tilde M$ must be isometric to one of the following spaces $\mathbb{S}^2$, $\CP^2$, $\HP^2$, or $\OP^2$.
Since the injectivity radius of $M$ is $\tfrac\pi2$, 
it has to be isometric to $\RP^2$, $\CP^2$, $\HP^2$, or $\OP^2$.

Denote by $M'\subset \RR^d$ the image of the corresponding Veronese embedding provided by the proposition.
Without loss of generality, we can assume that $d>26$, so $M'$ exists.

Recall that the second fundamental form $\II$ is a bilinear symmetric form on the tangent space with values in the normal space.
Assume there is a point $p\z\in M\cap M'$ such that $\T_pM=\T_pM'$ and the second fundamental forms of $M$ and $M'$ at $p$ coincide.
Then $M'\z=M$.
Indeed, since every geodesic is mapped to a round circle, the image of a geodesic in direction ${\mathsc{u}}\in \T_p$ is completely described by $\II({\mathsc{u}},{\mathsc{u}})$.
And these circles sweep the whole $M$ and $M'$.

Recall that the extrinsic curvature tensor $\Phi$ is defined as
\[\Phi({\mathsc{x}},{\mathsc{y}},{\mathsc{v}},{\mathsc{w}})=\langle \II({\mathsc{x}},{\mathsc{y}}),\II({\mathsc{v}},{\mathsc{w}})\rangle;\]
see \cite{petrunin2003}.
Note that it describes the second fundamental form $\II$ up to motion of the ambient space.
Therefore, once we show that the extrinsic curvature tensors of $M$ and $M'$ coincide at one point,
we get that $M$ and $M'$ are congruent.

The tensor $\Phi$ can be written as
\[\Phi({\mathsc{x}},{\mathsc{y}},{\mathsc{v}},{\mathsc{w}})=E({\mathsc{x}},{\mathsc{y}},{\mathsc{v}},{\mathsc{w}})+\tfrac 1 3\cdot(\Rm({\mathsc{x}},{\mathsc{v}},{\mathsc{y}},{\mathsc{w}})+\Rm({\mathsc{x}},{\mathsc{w}},{\mathsc{y}},{\mathsc{v}}))\]
where $E$ is the total symmetrization of $\Phi$; that is,
$$E({\mathsc{x}},{\mathsc{y}},{\mathsc{v}},{\mathsc{w}})=\tfrac 1 3\cdot
(\Phi({\mathsc{x}},{\mathsc{y}},{\mathsc{v}},{\mathsc{w}})+\Phi({\mathsc{y}},{\mathsc{v}},{\mathsc{x}},{\mathsc{w}})+\Phi({\mathsc{v}},{\mathsc{x}},{\mathsc{y}},{\mathsc{w}})),$$
and
$$\Rm({\mathsc{x}},{\mathsc{y}},{\mathsc{v}},{\mathsc{w}})=\Phi({\mathsc{x}},{\mathsc{v}},{\mathsc{y}},{\mathsc{w}})-\Phi({\mathsc{x}},{\mathsc{w}},{\mathsc{y}},{\mathsc{v}})$$
is the Riemannian curvature tensor of $M$.

Since $M$ is isometric to $M'$, they have the same Riemannian curvature tensors.
It remains to show that the $E$-tensors are the same.
But 
\[f({\mathsc{x}})=E({\mathsc{x}},{\mathsc{x}},{\mathsc{x}},{\mathsc{x}})\z=|\II({\mathsc{x}},{\mathsc{x}})|^2\]
is a homogeneous polynomial of degree $4$ on the tangent space 
and it describes $E$ completely.
Since all normal curvatures are equal to 2,
we get $\II({\mathsc{x}},{\mathsc{x}})\z=2\cdot|{\mathsc{x}}|^2$ and $E({\mathsc{x}},{\mathsc{x}},{\mathsc{x}},{\mathsc{x}})=4\cdot|{\mathsc{x}}|^4$ 
for both embeddings and for any tangent vector ${\mathsc{x}}$.
This finishes the proof.
\qeds

\paragraph{Final remarks.}
This note was motivated by the following question \cite{petrunin2023}.
Recall that Veronese embedding maps $\RP^n$, $\CP^n$, and $\HP^n$ into balls of radius $r_n= \sqrt{n/(2\cdot n+2)}$

\begin{thm}{Question}
Is it true that the Veronese embedding minimizes the maximal normal curvature among all smooth embeddings of $\RP^n$ into the ball of radius $r_n$ in a Euclidean space of large dimension?
\end{thm}

The same question can also be asked about $\CP^n$ and $\HP^n$. 
A keen reader might have noticed that the case $n=2$ is already solved.

\begin{thm}{Question}
Let $M$ be as in our sphere theorem;
does it have to be diffeomorphic to the standard $n$-sphere?
\end{thm}


I suspect that the answer is ``yes''.
If, in addition, $M$ lies in the boundary of the $r$-ball, then by the Gauss formula \cite[Lemma 5]{petrunin2024}, its sectional curvature is strictly quarter-pinched,
and in this case, it has to be diffeomorphic to a sphere~\cite{brendle-schoen}.

\paragraph{Acknowledgments.}
I want to thank Alexander Lytchak for help.
This work was partially supported by the National Science Foundation, grant DMS-2005279.

{\sloppy
\def\emph{\textit}
\printbibliography[heading=bibintoc]
\fussy
}
\end{document}

The Tohoku Mathematical Journal. Second Series
