\documentclass[a4paper,10pt]{article}
\usepackage{paper-en}
\usepackage{hyperref}




%\usepackage[notref,notcite,color]{showkeys}



%\def\thetitle{Mildly curved submanifolds in a ball}
\def\thetitle{Veronese minimizes normal curvatures}
\def\theauthors{Anton Petrunin}

\hypersetup{colorlinks=true,
citecolor=black,
linkcolor=black,
anchorcolor=black,
filecolor=black,
menucolor=black,
urlcolor=black,
pdftitle={\thetitle},
pdfauthor={\theauthors}
}








%\usepackage[a-2b,mathxmp]{pdfx}[2018/12/22]
%\overfullrule=100mm
%\usepackage[none]{hyphenat}
\begin{document}
%\pagestyle{empty}\renewcommand\includegraphics[2][{}]{}


\title{\thetitle}
\author{\theauthors}
\date{}
\maketitle

\begin{abstract}
Suppose $M$ is a closed submanifold in a Euclidean ball of large dimension.
We give an optimal bound on the normal curvatures that guarantee that $M$ is a sphere.
The border cases consist of Veronese embeddings of the four projective planes.
\end{abstract}

\paragraph{Introduction.} Let $M\subset \RR^d$ be a closed smooth $n$-dimensional submanifold.
Assume $d$ is large and $M$ lies in an $r$-ball.
\textit{What can we say about the normal curvatures of $M$?}

First note that the curvatures cannot be smaller than $\tfrac1r$ at all points.
Moreover, 
\textit{the average value of $|H|$ must be at least $n\cdot\tfrac1r$};
here $H$ denotes the mean curvature vector \cite[28.2.5]{burago-zalgaller}, \cite[3.1]{petrunin2024a}.
This statement is a straightforward generalization of the result of István Fáry about closed curves in a ball \cite{fary,tabachnikov}.

On the other hand, the $n$-dimensional torus can be embedded into an $r$-ball with all normal curvatures $\sqrt{3\cdot n/(n+2)}\cdot\tfrac1r$.
This embedding was found by Michael Gromov
\cite[2.A]{gromov3}, \cite[1.1.A]{gromov2}.
The bound is optimal; that is, any smooth $n$-dimensional torus in an $r$-ball has normal curvature at least $\sqrt{3\cdot n/(n+2)}\cdot\tfrac1r$ at some point
\cite{petrunin2024a}.
Gromov's examples easily imply the following:
any closed smooth manifold $M$ admits a smooth embedding into an $r$-ball of sufficiently large dimension with normal curvatures less than $\sqrt{3}\cdot\tfrac1r$
\cite[1.D]{gromov3}, \cite[1.1.C]{gromov2}.
\textit{But what happens between $\tfrac1r$ and  $\sqrt{3}\cdot\tfrac1r$?}

In this note, we consider embeddings in an $r$-ball with normal curvatures at most $\tfrac2{\sqrt{3}}\cdot \tfrac1r$.
We show that if the inequality is strict, then the manifold must be homeomorphic to a sphere (see §~\ref{thm:strict}).
For the nonstrict inequality, in addition to spheres, we get real, complex, quaternionic, and octonionic planes mapped by rescaled Veronese embeddings (see §~\ref{thm:=}).

\paragraph{Sphere theorem.}
\label{thm:strict}
\textit{Let $M$ be a closed smooth $n$-dimensional submanifold in a closed $r$-ball in $\RR^d$.
Suppose that the normal curvatures of $M$ are strictly less than $\tfrac2{\sqrt{3}}\cdot\tfrac1r$.
Then $M$ is homeomorphic to the $n$-sphere.}


\parit{Proof.}
We can assume that $n\ge 2$;
otherwise there is nothing to prove.

Denote the $r$-ball by $\BB^d$.
We can assume that $r=\tfrac1{\sqrt{3}}$; that is, $r$ is the circumradius of an equilateral triangle with side 1.
Therefore, the normal curvatures of $M$ are smaller than $2$.

Choose a unit-speed geodesic $\gamma\:[0,\tfrac\pi2]\to M$;
let $x=\gamma(0)$ and $y=\gamma(\tfrac\pi2)$.
By the assumption, the curvature of $\gamma$ in $\RR^d$ is less than~$2$.
Applying Schur's bow lemma, we get $|x-y|>1$.

Let $\Pi$ be the perpendicular bisector to $[x,y]$.
Since the curvature of $\gamma$ is smaller than 2,
\[\measuredangle(\gamma'(t_0),\gamma'(t))< 2\cdot|t-t_0|,
\quad\text{and}\quad
\langle \gamma'(t_0),\gamma'(t) \rangle> \cos (2\cdot|t-t_0|)\] for $t\ne t_0$.
Therefore,
\[\langle y-x,\gamma'(t_0) \rangle>\int\limits_0^{\frac\pi2}\cos (2\cdot |t-t_0|)\cdot dt\ge0.\]
In particular, the derivative of function 
$f\:t\mapsto \langle y-x,\gamma(t) \rangle$
is positive.
Therefore, $\gamma$ intersects $\Pi$ transversely at a single point;
denote it by $s$.



Choose a unit vector $\mathsc{u}\in\T_x$;
let $\gamma_{\mathsc{u}}\:[0,\tfrac\pi2]\to M$ be the unit-speed geodesic that starts from $x$ in the direction ${\mathsc{u}}$, and let $z=\gamma_{\mathsc{u}}(\tfrac\pi2)$.
The argument above shows that $|x-z|>1$.

Denote by $H_x$ and $H_y$ the closed half-spaces bounded by $\Pi$ that contain $x$ and $y$ respectively.
Assume $z\in H_x$, then we have $|y-z|\ge |x-z|>1$.
Since $|x-y|>1$, the triangle $[xyz]$ has all sides larger than $1$,
which is impossible since $x,y,z\in \BB^d$.
Therefore, $\gamma_{\mathsc{u}}$ meets $\Pi$ before in $\tfrac\pi2$;
denote by $r({\mathsc{u}})$ be the first such time moment.

\begin{wrapfigure}{r}{36 mm}
\vskip-4mm
\centering
\includegraphics{mppics/pic-10}
\vskip0mm
\end{wrapfigure}

Let us show that the function ${\mathsc{u}}\mapsto r({\mathsc{u}})$ is smooth.
In other words, $\gamma_{\mathsc{u}}$ intersects $\Pi$ transversely at time $r({\mathsc{u}})$.
Assume this is not the case, so $\gamma_{\mathsc{u}}$ is tangent to $\Pi$ at $r({\mathsc{u}})$.
Let $\hat\gamma_{\mathsc{u}}$ be the concatenation of the reflection of $\gamma_{\mathsc{u}}|_{[0,r({\mathsc{u}})]}$ across $\Pi$ and $\gamma_{\mathsc{u}}|_{[r({\mathsc{u}}),\frac\pi2]}$.
Note that $\hat \gamma_{\mathsc{u}}$ is $C^1$-smooth, and it is $C^\infty$-smooth everywhere except $r({\mathsc{u}})$.
Therefore, Schur's bow lemma is applicable to~$\hat \gamma_{\mathsc{u}}$, and hence, $|y-z|>1$.
Again, all sides of triangle $[xyz]$ are larger than $1$;
hence, it cannot lie in $\BB^d$ --- a contradiction.  

It follows that the set 
\[V_x=\set{t\cdot {\mathsc{u}}\in \T_x}{|{\mathsc{u}}|=1,\quad 0\le t\le r({\mathsc{u}}),}\]
is diffeomorphic to the closed $n$-disc.
Denote by $W_x$ the connected component of $x$ in $M\cap H_x$.

From the Gauss formula \cite[Lemma 5]{petrunin2024}, the sectional curvatures of $M$ are less than $4$.
In particular, the exponential map $\exp_x\:\T_x\to M$ is a local diffeomorphism in the $\tfrac\pi2$-ball centered at the origin of $\T_x$.

It follows that $\exp_x\:V_x\to W_x$ is a local diffeomorphism;
in particular, $W_x$ is a smooth manifold with boundary.
Since $V_x$ is simply connected, $\exp_x$ defines a diffeomorphism $V_x\to W_x$.
In particular, $W_x$ is a closed topological $n$-disc and $\partial W_x$ is a smooth hypersurface in $M$.

Let us swap the roles of $x$ and $y$, and repeat the construction.
We get another closed topological $n$-disc $W_y\subset M$ bounded by a smooth hypersurface $\partial W_y$.

Observe that $\partial W_x$ intersects $\partial W_y$ at $s$.
Furthermore, both $\partial W_x$ and $\partial W_y$ are connected components of $s$ in $M\cap \Pi$.
Therefore, $\partial W_x=\partial W_y$.
That is, $M$ can be obtained by gluing two $n$-discs by a diffeomorphism between their boundaries.
Hence $M$ is homeomorphic to the $n$-sphere.
\qeds

\paragraph{Veronese embeddings.}\label{veronese}
The real, complex, quaternionic projective spaces of dimension $n$, and the octonionic projective plane
will be denoted by $\RP^n$, $\CP^n$, $\HP^n$, and $\OP^2$ respectively.
We assume that each of these spaces is equipped with the canonical metric;
in particular, all the spaces have closed geodesics of length~$\pi$.

\begin{thm}{Proposition}
There are smooth isometric embeddings
\begin{itemize}
 \item $\RP^n \hookrightarrow\RR^d$ for $d\ge n+\tfrac12\cdot n\cdot(n+1)$;
 \item $\CP^n \hookrightarrow\RR^d$ for $d\ge n+n\cdot(n+1)$;
 \item $\HP^n \hookrightarrow\RR^d$ for $d\ge n+2\cdot n\cdot(n+1)$;
 \item $\OP^2 \hookrightarrow\RR^d$ for $d\ge 26$;
\end{itemize}
that map each geodesic to a round circle.

Moreover,

\begin{subthm}{2}
all normal curvatures of the images of these embeddings are equal to $2$
\end{subthm}

\begin{subthm}{r}
the images of these embeddings lie in a sphere of radius $r_n=\sqrt{n/(2\cdot n+2)}$ (for $\OP^2$, we assume that $n=2$).
\end{subthm}

\end{thm}

The proposition can be extracted from two theorems in \cite[§ 2]{sakamoto}.
The embeddings provided by the proposition will be called \emph{Veronese embeddings}.
Note that $r_n$ is the circumradius of a regular $n$-simplex with edge length~$1$.
Note that $r_2=1/\sqrt{3}$; therefore, the second part of proposition shows that our sphere theorem has the optimal bound. 

The Veronese embeddings have a very explicit algebraic description and many nice geometric properties.
In particular,
these embeddings are equivariant, and 
their images are minimal submanifolds in the $r_n$-spheres.
All of this is discussed in the cited paper by Kunio Sakamoto.

The following lemma is also closely related to the result of Kunio Sakamoto;
it implies that the properties in the proposition uniquely describe
Veronese embeddings up to motion of the ambient space.

\begin{thm}{Lemma}
Let $M$ and $M'$ be intrinsically isometric smooth submanifolds in $\RR^d$.
Suppose that all geodesics in $M$ and $M'$ are closed
and each geodesic forms a round circle in $\RR^d$.
Then $M$ is congruent to $M'$;
that is, there is an isometry of $\RR^d$ that maps $M$ to $M'$.
\end{thm}

\parit{Proof.}
Recall that the second fundamental form $\II$ of a submanifold is a bilinear symmetric form on the tangent space with values in the normal space.
Assume there is a common point $p$ on $M$ and $M'$ 
with common tangent space $\T_pM\z=\T_pM'$ and such that the second fundamental forms of $M$ and $M'$ at $p$ coincide.
Then $M'\z=M$.
Indeed, since every geodesic is mapped to a round circle, the image of a geodesic in direction ${\mathsc{u}}\in \T_p$ is completely described by $\II({\mathsc{u}},{\mathsc{u}})$.
And these circles sweep the whole $M$ and $M'$.

Recall that the extrinsic curvature tensor $\Phi$ of a submanifold is defined as
\[\Phi({\mathsc{x}},{\mathsc{y}},{\mathsc{v}},{\mathsc{w}})=\langle \II({\mathsc{x}},{\mathsc{y}}),\II({\mathsc{v}},{\mathsc{w}})\rangle,\]
here $\mathsc{x}$, $\mathsc{y}$, $\mathsc{v}$, $\mathsc{w}$ are tangent vectors to the submanifold at some point; 
see \cite{petrunin2003}.
Note that the $\Phi$-tensor describes the second fundamental form $\II$ up to motion of the ambient space.
Therefore, once we show that the $\Phi$-tensors of $M$ and $M'$ coincide at one point,
we get that $M$ and $M'$ are congruent.

The tensor $\Phi$ can be written as
\[\Phi({\mathsc{x}},{\mathsc{y}},{\mathsc{v}},{\mathsc{w}})
=
\E({\mathsc{x}},{\mathsc{y}},{\mathsc{v}},{\mathsc{w}})+\tfrac 1 3\cdot(\Rm({\mathsc{x}},{\mathsc{v}},{\mathsc{y}},{\mathsc{w}})+\Rm({\mathsc{x}},{\mathsc{w}},{\mathsc{y}},{\mathsc{v}}))\]
where $\E$ is the total symmetrization of $\Phi$; that is,
$$\E({\mathsc{x}},{\mathsc{y}},{\mathsc{v}},{\mathsc{w}})=\tfrac 1 3\cdot
(\Phi({\mathsc{x}},{\mathsc{y}},{\mathsc{v}},{\mathsc{w}})+\Phi({\mathsc{y}},{\mathsc{v}},{\mathsc{x}},{\mathsc{w}})+\Phi({\mathsc{v}},{\mathsc{x}},{\mathsc{y}},{\mathsc{w}})),$$
and
$$\Rm({\mathsc{x}},{\mathsc{y}},{\mathsc{v}},{\mathsc{w}})=\Phi({\mathsc{x}},{\mathsc{v}},{\mathsc{y}},{\mathsc{w}})-\Phi({\mathsc{x}},{\mathsc{w}},{\mathsc{y}},{\mathsc{v}})$$
is the Riemannian curvature tensor of $M$.

Since $M$ is isometric to $M'$, they have the same Riemannian curvature tensors.
It remains to show that the $\E$-tensors are the same.
But 
\[f({\mathsc{x}})=\E({\mathsc{x}},{\mathsc{x}},{\mathsc{x}},{\mathsc{x}})\z=|\II({\mathsc{x}},{\mathsc{x}})|^2\]
is a homogeneous polynomial of degree $4$ on the tangent space 
and it describes $\E$ completely.

The geodesics in $M$ and $M'$ are closed and have the same length.
Since each of these geodesic forms a circle in $\RR^d$,
all these circles have the same curvature, say $\kappa$.
Therefore, $\II({\mathsc{x}},{\mathsc{x}})\z=\kappa\cdot|{\mathsc{x}}|^2$ and $\E({\mathsc{x}},{\mathsc{x}},{\mathsc{x}},{\mathsc{x}})=\kappa^2\cdot|{\mathsc{x}}|^4$ 
for both submanifolds and for any tangent vector ${\mathsc{x}}$.
This finishes the proof.
\qeds


\paragraph{Rigidity theorem.}\label{thm:=}
\textit{Let $M$ be a closed smooth $n$-dimensional submanifold in a closed $r$-ball in $\RR^d$.
Suppose that the normal curvatures of $M$ are at most $\tfrac2{\sqrt{3}}\cdot\tfrac1r$.
If $M$ is not homeomorphic to a sphere, then up to rescaling, it is congruent to an image of the Veronese embedding of a projective plane $\RP^2$, $\CP^2$, $\HP^2$, or $\OP^2$.}

\medskip


This result is an application of the following theorem;
its weaker form was proved by Detlef Gromoll and Karsten Grove \cite{gromoll-grove}, and
the final step was made by Burkhard Wilking \cite{wilking}.

\begin{thm}{Gromoll--Grove--Wilking theorem}\label{thm:GGW}
Let $M$ be a compact Riemannian manifold with sectional curvature at least $1$ and
diameter at least $\tfrac\pi2$.
If $M$ is not homeomorphic to a sphere, then its Riemannian universal cover is isometric to a compact rank-one symmetric space.
\end{thm}

Recall that a compact rank-one symmetric space is isometric to a rescaled copy of one of the following spaces:
$\RP^n$, $\CP^n$, $\HP^n$, $\OP^2$, and unit spheres $\mathbb{S}^n$; see for example, \cite[8.12.2]{wolf}.
As before we assume that these spaces are equipped with the canonical metrics;
in particular, all the projective spaces have closed geodesics of length~$\pi$.

\parit{Proof of the rigidity theorem.}
Assume $M$ is not homeomorphic to a sphere;
in this case, $n\ge 2$.
As before, $\BB^d$ will denote the $r$-ball in $\RR^d$, and we assume that $r=\tfrac1{\sqrt{3}}$;
therefore, the normal curvatures of $M$ are at most $2$.

By the proposition in §~\ref{veronese}, the images of Veronese embedding satisfy the assumption of the theorem.
It remains to show that there are no other embeddings of that type.


Choose a unit-speed geodesic $\gamma\:[0,\tfrac\pi2]\to M$;
let $x=\gamma(0)$ and $y=\gamma(\tfrac\pi2)$.
The argument in our sphere theorem implies that $|x-y|=1$.
The rigidity case in the bow lemma implies that $\gamma$ is a half-circle of curvature $2$.
Since any two points in $M$ can be connected by a geodesic, we get the following.
\begin{itemize}
 \item The diameter of $M$ is 1.
 \item The intrinsic diameter and injectivity radius of $M$ are equal to $\tfrac\pi2$.
 \item All geodesics in $M$ are circles of curvature 2 in $\RR^d$.
\end{itemize}

Furthermore, for $x$ and $y$ as above,
there is another point $z\in M$ such that $|x-z|=|y-z|=1$.
If not, then again, the argument in the sphere theorem would imply that $M$ is a sphere.
But since $x,y,z\in\BB^d$,
the equalities $|x-y|=|y-z|=|x-z|=1$ imply that $x\in \partial \BB^d$.

The choice of $x\in M$ was arbitrary.
Therefore, $M$ lies in the sphere $\partial \BB^d$ of radius $r=1/\sqrt{3}$.
This sphere has sectional curvature $1/r^2=3$;
all normal curvatures of $M$ in the sphere are $\kappa=\sqrt{2^2-1/r^2}=1$.
By the Gauss formula \cite[Lemma 5]{petrunin2024}, the sectional curvatures of $M$ are at least $3-2\cdot \kappa^2=1$.

By the Gromoll--Grove--Wilking theorem, the universal cover $\tilde M$ of $M$ is isometric to a rank-one symmetric space.
Taking into account the injectivity radius and curvature of $M$, we get that $\tilde M$ must be isometric to one of the following spaces
$\tfrac12\cdot \SSS^n$, $\SSS^n$, $\CP^n$, $\HP^n$ for some $n$, or $\OP^2$.
Note that the points $x,y,z\in M$ constructed above lie at an intrinsic distance $\tfrac\pi2$ from each other.
It forbids $\tfrac12\cdot \SSS^n$ for every $n$.
Furthermore, if $n\ge 3$, then each space  $\SSS^n$, $\CP^n$ and $\HP^n$ contain 4 points at a distance $\tfrac\pi2$ from each other.
Since the injectivity radius of $M$ is $\tfrac\pi2$, their projections in $M$ must lie at a distance $\tfrac\pi2$ from each other as well.
It follows that $\BB^d$ must contain 4 points at a distance 1 from each other, which is impossible.

Hence, $\tilde M$ must be isometric to one of the following spaces $\mathbb{S}^2$, $\CP^2$, $\HP^2$, or $\OP^2$.
Since the injectivity radius of $M$ is $\tfrac\pi2$, 
it has to be isometric to $\RP^2$, $\CP^2$, $\HP^2$, or $\OP^2$.

Denote by $M'\subset \RR^d$ the image of the corresponding Veronese embedding provided by the proposition in §~\ref{veronese}.
Without loss of generality, we can assume that $d$ is large, so $M'$ exists.
Applying the lemma in §~\ref{veronese}, we get that $M$ is congruent $M'$ --- hence the result. 
\qeds

\paragraph{Final remarks.}
Recall that the Veronese embeddings map $\RP^n$, $\CP^n$, and $\HP^n$ into balls of radius $r_n= \sqrt{n/(2\cdot n+2)}$, which is the circumradius of a regular $n$-simplex with edge length~$1$.
This note is motivated by the following question~\cite{petrunin2023}.

\begin{thm}{Question}
Is it true that the Veronese embedding minimizes the maximal normal curvature among all smooth embeddings of $\RP^n$ into the ball of radius $r_n$ in a Euclidean space of large dimension?
\end{thm}

The same question can be asked about $\CP^n$ and $\HP^n$. 
A keen reader might have noticed that the case $n=2$ is already solved.

\begin{thm}{Question}
Let $M$ be as in our sphere theorem;
does it have to be diffeomorphic to the standard $n$-sphere?
\end{thm}

I suspect that the answer is \textit{yes}.
If, in addition, $M$ lies on the boundary of the $r$-ball, then by the Gauss formula \cite[Lemma 5]{petrunin2024}, $M$  has strictly quarter-pinched curvature;
so, it has to be diffeomorphic to a standard sphere~\cite{brendle-schoen}.

\paragraph{Acknowledgments.}
I want to thank Alexander Lytchak for help.
This work was partially supported by the National Science Foundation, grant DMS-2005279.

{\sloppy
\def\emph{\textit}
\printbibliography[heading=bibintoc]
\fussy
}
\end{document}

The Tohoku Mathematical Journal. Second Series
