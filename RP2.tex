\documentclass[a4paper,10pt]{article}
\usepackage{paper-en}
\usepackage{hyperref}




%\usepackage[notref,notcite,color]{showkeys}


\def\thetitle{Mildly curved submanifolds in a ball}
\def\theauthors{Anton Petrunin}

\hypersetup{colorlinks=true,
citecolor=black,
linkcolor=black,
anchorcolor=black,
filecolor=black,
menucolor=black,
urlcolor=black,
pdftitle={\thetitle},
pdfauthor={\theauthors}
}








%\usepackage[a-2b,mathxmp]{pdfx}[2018/12/22]

%\overfullrule=100mm

\begin{document}
%\pagestyle{empty}\renewcommand\includegraphics[2][{}]{}


\title{\thetitle}
\author{\theauthors}
\date{}
\maketitle

\begin{abstract}
Suppose $M$ is a closed submanifold in a Euclidean ball of large dimension.
We give an optimal bound on the normal curvatures of $M$ that guarantee that it is a sphere.
The border cases consist of Veronese embeddings of four projective planes.
\end{abstract}

\paragraph{Introduction.} Let $M\subset \RR^d$ be a closed smooth $n$-dimensional submanifold.
Assume $d$ is large and $M$ lies in an $r$-ball.
\textit{What can we say about the normal curvatures of $M$?}

First note that the curvatures cannot be smaller than $\tfrac1r$ at all points.
Moreover, 
\textit{the average value of $|H|$ must be at least $n\cdot\tfrac1r$};
here $H$ denotes the mean curvature vector \cite[28.2.5]{burago-zalgaller}, \cite[3.1]{petrunin2024a}.
This statement is a straightforward generalization of the result of István Fáry \cite{fary,tabachnikov} about closed curves in a ball.

On the other hand, the $n$-dimensional torus can be embedded into an $r$-ball with all normal curvatures $\sqrt{3\cdot n/(n+2)}\cdot\tfrac1r$.
This embedding was found by Michael Gromov
\cite[2.A]{gromov3}, \cite[1.1.A]{gromov2}.
This bound is optimal; that is, any smooth $n$-dimensional torus in an $r$-ball has normal curvature at least $\sqrt{3\cdot n/(n+2)}\cdot\tfrac1r$ at some point
\cite{petrunin2024a}.
Gromov's examples easily imply the following:
any closed smooth manifold $M$ admits a smooth embedding into an $r$-ball of sufficiently large dimension with normal curvatures less than $\sqrt{3}\cdot\tfrac1r$
\cite[1.D]{gromov3}, \cite[1.1.C]{gromov2}.
\textit{But what happens between $\tfrac1r$ and  $\sqrt{3}\cdot\tfrac1r$?}

In this note, we consider embeddings in an $r$-ball with normal curvatures at most $\tfrac2{\sqrt{3}}\cdot \tfrac1r$.
We show that if the inequality is strict, then the manifold must be homeomorphic to a sphere (see §~\ref{thm:strict}).
For the nonstrict inequality, in addition to spheres we get real, complex, quaternionic, and octonionic planes mapped by the corresponding Veronese embedding up to rescaling (see §~\ref{thm:=}).

\paragraph{Sphere theorem.}
\label{thm:strict}
\textit{Let $M$ be a closed smooth $n$-dimensional submanifold in a closed $r$-ball in $\RR^d$.
Suppose that the normal curvatures of $M$ are strictly less than $\tfrac2{\sqrt{3}}\cdot\tfrac1r$.
Then $M$ is homeomorphic to the $n$-sphere.}


\parit{Proof.}
Denote the $r$-ball by $\BB^d$.
We can assume that $r=\tfrac1{\sqrt{3}}$;
therefore, the normal curvatures of $M$ are smaller than $2$.

Choose a unit-speed geodesic $\gamma\:[0,\tfrac\pi2]\to M$;
let $x=\gamma(0)$ and $y=\gamma(\tfrac\pi2)$.
By the assumption, the curvature of $\gamma$ in $\RR^d$ is less than~$2$.
Applying Schur's bow lemma, we get $|x-y|>1$.

Let $\Pi$ be the perpendicular bisector to $[x,y]$.
Since the curvature of $\gamma$ is smaller than 2,
\[\measuredangle(\gamma'(t_0),\gamma'(t))< 2\cdot|t-t_0|,
\quad\text{and}\quad
\langle \gamma'(t_0),\gamma'(t) \rangle> \cos (2\cdot|t-t_0|)\] if $|t-t_0|>0$.
Therefore
\[\langle y-x,\gamma'(t_0) \rangle>\int\limits_0^{\frac\pi2}\cos (2\cdot |t-t_0|)\cdot dt\ge0.\]
In particular, the function $f\:[0,\tfrac\pi2]\to\RR$ defined by
$f\:t\mapsto \langle y-x,\gamma(t) \rangle$
has positive derivative.
Therefore, $\gamma$ intersects $\Pi$ transversely at a single point;
denote it by $s$.



Choose a unit vector $u\in\T_x$;
let $\gamma_u\:[0,\tfrac\pi2]\to M$ be the unit-speed geodesic that starts from $x$ in the direction $u$, and let $z=\gamma_u(\tfrac\pi2)$.
The argument above shows that $|x-z|>1$.

Denote by $H_x$ and $H_y$ the closed half-spaces bounded by $\Pi$ that contain $x$ and $y$ respectively.
Assume $z\in H_x$, then we have $|y-z|\ge |x-z|>1$.
Since $|x-y|>1$, the triangle $[xyz]$ has all sides larger than $1$,
which is impossible since $x,y,z\in \BB^d$.
Therefore, $\gamma_v$ meets $\Pi$ before in $\tfrac\pi2$;
denote by $r(u)$ be the first such time moment.

\begin{wrapfigure}{r}{44 mm}
\vskip-0mm
\centering
\includegraphics{mppics/pic-10}
\vskip2mm
\end{wrapfigure}

Let us show that the function $u\mapsto r(u)$ is smooth.
In other words, $\gamma_u$ intersects $\Pi$ transversely at time $r(u)$.
Assume this is not the case, so $\gamma_u$ is tangent to $\Pi$ at $r(u)$.
Let $\hat\gamma_u$ be the concatenation of the reflection of $\gamma_u|_{[0,r(u)]}$ across $\Pi$ and $\gamma_u|_{[r(u),\frac\pi2]}$.
Note that $\hat \gamma_u$ is $C^1$-smooth and it is $C^\infty$-smooth everywhere except $r(u)$.
Therefore, Schur's bow lemma is applicable to~$\hat \gamma_u$, and hence, $|y-z|>1$.
Again, all sides of triangle $[xyz]$ are larger than $1$;
hence, it cannot lie in $\BB^d$ --- a contradiction.  

It follows that the set 
\[V_x=\set{t\cdot v\in \T_x}{|u|=1,\quad 0\le t\le a(v),}\]
is diffeomorphic to the closed $n$-disc.
Denote by $W_x$ the connected component of $x$ in $M\cap H_x$.

From the Gauss formula \cite[Lemma 5]{petrunin2024}, the sectional curvatures of $M$ are less than $4$.
In particular, the exponential map $\exp_x\:\T_x\to M$ is a local diffeomorphism in the $\tfrac\pi2$-ball centered at the origin of $\T_x$.

It follows that $\exp_x\:V_x\to W_x$ is a local diffeomorphism;
in particular, $W_x$ is a smooth manifold with boundary.
Since $V_x$ is simply connected, $\exp_x$ defines a diffeomorphism $V_x\to W_x$.
In particular, $W_x$ is a closed topological $n$-disc and $\partial W_x$ is a smooth hypersurface in $M$.

Let us swap the roles of $x$ and $y$, and repeat the construction.
We get another closed topological $n$-disc $W_y\subset M$ bounded by a smooth hypersurface $\partial W_y$.

Observe that $\partial W_x$ intersects $\partial W_y$ at $s$.
Furthermore, both $\partial W_x$ and $\partial W_y$ are connected components of $s$ in $M\cap \Pi$.
Therefore, $\partial W_x=\partial W_y$.
That is, $M$ can be obtained by gluing two $n$-discs by a diffeomorphism between their boundaries.
Hence $M$ is homeomorphic to the $n$-sphere.
\qeds

\paragraph{Veronese embeddings.}\label{thm:=}
The real, complex, quaternionic, and octonionic projective planes will be denoted by
$\RP^2$, $\CP^2$, $\HP^2$, and $\OP^2$, respectively.
We assume that each of these planes is equipped with the canonical metric;
it has closed geodesics of length $\pi$ and therefore its sectional curvature is in the range $[1,4]$.

\begin{thm}{Proposition}
There are smooth isometric embeddings
\begin{align*}
\RP^2& \hookrightarrow\RR^5 , 
&
\CP^2& \hookrightarrow\RR^8,
&
\HP^2& \hookrightarrow\RR^{14}, 
&
\OP^2& \hookrightarrow\RR^{26}
\end{align*}
that maps each geodesic to a round circle. 
\end{thm}

This proposition can be extracted from two theorems in \cite[§ 2]{sakamoto}.
The embeddings provided by the proposition will be called \emph{Veronese embeddings}.
From the proof below it will follow that the properties in the proposition uniquely describe
Veronese embeddings up to isometric inclusion.

The Veronese embeddings have a very explicit algebraic description and many nice geometric properties.
In particular,
these embeddings are equivariant,
their images lie in spheres of radius $1/\sqrt3$,
and they are minimal submanifolds in these spheres.
All of this is discussed in the cited paper by Kunio Sakamoto.
 

 

\begin{thm}{Rigidity theorem}
Let $\BB^d$ be $\tfrac1{\sqrt{3}}$ ball in $\RR^d$.
Suppose a closed manifold $M$ admits a smooth embedding $f$ with normal curvatures at most $2$ in $\BB^d$ for some~$d$.
If $M$ is not diffeomorphic to a sphere, then it is isometric to $\RP^2$, $\CP^2$, $\HP^2$, or $\OP^2$.
Moreover, $f$ can be written as one of the following compositions
\begin{align*}
\RP^2& \hookrightarrow\RR^5 \hookrightarrow\RR^d, 
&
\CP^2& \hookrightarrow\RR^8 \hookrightarrow\RR^d,
\\
\HP^2& \hookrightarrow\RR^{14} \hookrightarrow\RR^d, 
&
\OP^2& \hookrightarrow\RR^{26} \hookrightarrow\RR^d;
\end{align*}
in each composition the first map is the Veronese embedding, and the second map is an isometric inclusion. 


\end{thm}

The proof is an application of the following theorem.
In a weaker form,
it was proved by Detlef Gromoll and Karsten Grove \cite{gromoll-grove};
the final step was made by Burkhard Wilking \cite{wilking}.

\begin{thm}{Gromoll--Grove--Wilking theorem}\label{thm:GGW}
Let $M$ be a compact Riemannian manifold with sectional curvature at least $1$ and
diameter at least $\tfrac\pi2$.
If $M$ is not homeomorphic to a sphere, then it is locally isometric to a rank-one symmetric space.
\end{thm}

\parit{Proof of the rigidity theorem.}
Assume $M$ is not a sphere.

Choose a unit-speed geodesic $\gamma\:[0,\tfrac\pi2]\to M$;
let $x=\gamma(0)$ and $y=\gamma(\tfrac\pi2)$.
The argument in our sphere theorem implies that $|x-y|=1$.
The rigidity case in the bow lemma implies that $\gamma$ is a half-circle of curvature $2$.
Since any two points in $M$ can be connected by a geodesic, we get
\begin{itemize}
 \item the diameter of $M$ is 1,
 \item the intrinsic diameter and injectivity radius of $M$ are equal to $\tfrac\pi2$,
 \item all geodesics in $M$ are circles of curvature 2.
\end{itemize}

Furthermore, for $x$ and $y$ as above,
there is another point $z\in M$ such that $|x-z|=|y-z|=1$.
If not, then again, the argument in our sphere theorem would imply that $M$ is a sphere.
But since $x,y,z\in\BB^d$,
the equalities $|x-y|=|y-z|=|x-z|=1$ imply that $x\in \partial \BB^d$.

The choice of $x\in M$ was arbitrary.
Therefore, $M$ lies in the sphere $\partial \BB^d$ of radius $r=1/\sqrt{3}$.
This sphere has sectional curvature $1/r^2=3$;
the normal curvatures of $M$ in the sphere are $\sqrt{2^2-1/r^2}=1$.
By the Gauss formula \cite[Lemma 5]{petrunin2024}, the sectional curvatures of $M$ are at least $3-2\cdot 1^2=1$.

By the Gromoll--Grove--Wilking theorem, the universal cover $\tilde M$ of $M$ is isometric to a rank-one symmetric space.
Let us show that $\tilde M$ is either $\mathbb{S}^2$, $\CP^2$, $\HP^2$, or $\OP^2$.
Indeed, from above, the diameter of $\tilde M$ must be multiple of $\tfrac\pi2$ and it must have exactly 3 points on distance $\tfrac\pi2$; it excludes all other choices.
Since the injectivity radius of $M$ is $\tfrac\pi2$, 
it has to be isometric to $\RP^2$, $\CP^2$, $\HP^2$, or $\OP^2$.

We can assume that $d>26$;
denote by $v\:M\hookrightarrow \RR^d$ the corresponding Veronese embedding;
that is, $v$ is a composition of Veronese embedding and isometric inclusion $M\hookrightarrow \RR^m\hookrightarrow \RR^d$, where $m=5$, $8$, $14$, or $26$ for $\RP^2$, $\CP^2$, $\HP^2$, or $\OP^2$, respectively.

Recall that the second fundamental form is a bilinear symmetric form on tangent space with values in the normal space.
Assume that there is a motion $\iota$ of $\RR^d$ such that for some point $p\in M$ we have 
\[f(p)=\iota\circ v(p),\qquad \T_{f(p)}[f(M)]\z=\T_{f(p)}[\iota\circ v(M)],\]
and the second fundamental forms of $f$ and $v$ at $p$ coincide.
Then $f(M)\z=\iota\circ v(M)$.
Indeed, since every geodesic is mapped to a round circle, the image of a geodesic in direction $w\in \T_p$ is completely described by $\II(w,w)$.
And these circles sweep the image of the whole $M$.

Recall that extrinsic curvature tensor $\Phi$ is defined as
\[\Phi(X,Y,V,W)=\langle \II(X,Y),\II(V,W)\rangle;\]
see \cite{petrunin2003}.
Note that it describes the second fundamental form $\II$ up to rotation of the normal space.
Therefore, once we show that the extrinsic curvature tensor coincides for $f(M)$ and $v(M)$ at one point,
we get that $f(M)$ and $v(M)$ are congruent.

The tensor $\Phi$ can be written as
\[\Phi(X,Y,V,W)=E(X,Y,V,W)+\tfrac 1 3\cdot(\Rm(X,V,Y,W)+\Rm(X,W,Y,V))\]
where $E$ is the total symmetrization of $\Phi$; that is,
$$E(X,Y,V,W)=\tfrac 1 3\cdot
(\Phi(X,Y,V,W)+\Phi(Y,V,X,W)+\Phi(V,X,Y,W)),$$
and
$$\Rm(X,Y,V,W)=\Phi(X,V,Y,W)-\Phi(X,W,Y,V)$$
is the Riemannian curvature tensor of $M$.

Since both embeddings $v$ and $f$ are isometric, they induce the same Riemannian curvature tensor on $M$.
It remains to show that the $E$-tensors are the same.
But 
\[f(X)=E(X,X,X,X)\z=|\II(X,X)|^2\]
is a homogeneous polynomial of degree $4$ on the tangent space 
and it describes $E$ completely.
All normal curvatures are equal to 2,
so $\II(X,X)\z=2\cdot|X|^2$ and $E(X,X,X,X)=4\cdot|X|^4$ 
for both embeddings and for any $X$.
This finishes the proof.
\qeds

\paragraph{Final remarks.}
This note was motivated by the following question \cite{petrunin2023}.

\begin{thm}{Question}
Is it true that the Veronese embedding minimizes the maximal normal curvature among all smooth embeddings of $\RP^n$ into $r$-balls in Euclidean space of large dimension?
\end{thm}

The same question can also be asked about $\CP^n$ and $\HP^n$. 
A keen reader might have noticed that the case $n=2$ is already solved.

I suspect that the answer to the following question is ``yes''.

\begin{thm}{Question}
Let $M$ be as in our sphere theorem;
does it have to be diffeomorphic to the standard $n$-sphere?
\end{thm}

If, in addition, $M$ lies in the boundary of the $r$-ball, then by the Gauss formula \cite[Lemma 5]{petrunin2024}, its sectional curvature is strictly quarter-pinched;
in this case, it \textit{is} diffeomorphic to a sphere \cite{brendle-schoen}.

\paragraph{Acknowledgments.}
I want to thank Alexander Lytchak for help.
This work was partially supported by the National Science Foundation, grant DMS-2005279.

{\sloppy
\def\emph{\textit}
\printbibliography[heading=bibintoc]
\fussy
}
\end{document}

The Tohoku Mathematical Journal. Second Series
